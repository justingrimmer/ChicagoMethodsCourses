\documentclass{article}
\usepackage{amsmath}
\usepackage{amssymb}
\usepackage{url}

\usepackage[pdftex, bookmarksopen=true, bookmarksnumbered=true,
pdfstartview=FitH, breaklinks=true, urlbordercolor={0 1 1}, citebordercolor={0 1 0}, colorlinks = true]{hyperref}


\textheight 9in
\textwidth 6.5in
\topmargin -.5in
\oddsidemargin 0in
\evensidemargin 0in

\newcommand{\percent}{\ensuremath{\%\;}}



\begin{document}

\textbf{\huge{Content For Methods Comp}}

\section*{Probability Theory}


\subsection*{Books}
\begin{itemize}
\item[1)]  Bertsekas, Dimitri P. and Tsitsiklis, John (BT) Introduction to Probability Theory (second edition)
\item[2)] Ross, Sheldon.  First Course in Probability Theory (8th Edition).  
\end{itemize}




\section{A Rigorous Model of Probability and Its Properties}
\begin{itemize}
\item Axioms of probability
\item Inclusion/Exclusion
\item Conditional Probability
\item Bayes' rule
\item Independence 
\end{itemize}
\textit{Require Reading}
\begin{itemize}
\item[-] BT Chapter 1 
\end{itemize}



\section{Discrete  and Continuous Random Variables}
\begin{itemize}
\item Probability Mass Functions, Probability Density Functions
\item Cumulative Mass Functions, Cumulative Density Functions
\item Expectation, Variance
\item Mass functions/density functions for famous random variables
\item Functions of random variables. 
\end{itemize}
\textit{Required Reading}
\begin{itemize}
\item[-] BT Chapter 2 and 3
\end{itemize}




\section{Joint Distributions} 
\begin{itemize}
\item Joint mass functions, joint density functions
\item Marginalization
\item Independence of random variables
\item Expectation, covariance, variance
\end{itemize}
\textit{Required reading}
\begin{itemize}
\item[-] Handout (Chapter 6, Ross)
\end{itemize}

\section{Properties of Expectation, Moment Generating Functions, and Transformations}

\begin{itemize}
\item Change of coordinates
\item Properties of moment generating functions
\item Iterated expectations
\end{itemize}
textit{Require Readings}
\begin{itemize}
\item[-] BT Chapter 4
\end{itemize}






\section{Convergence, Limit Theorems, and Inequalities}

\begin{itemize}
\item Weak Law of Large Numbers
\item Convergence in Probability
\item Almost Sure Convergence
\item Convergence in Distribution
\end{itemize}
\textit{Required Readings}
\begin{itemize}
\item[-] BT Chapter 5
\end{itemize}

\clearpage

\setcounter{section}{0}
\section*{Regression}
\subsection*{Books}

\begin{itemize}	
%\item Bertsekas, Dimitri and Tsitsiklis, John. \emph{Introduction to Probability}. 2nd edition.
\item[] Regression:
\item Greene, William. \emph{Econometric Analysis}. 8th Edition.  
\item  Wooldridge, Jeffrey. \emph{Introductory Econometrics}. New York: South-Western. 5th edition.
\item R. Carter Hill, William E. Griffiths, and Guay C. Lim. \emph{Principles of Econometrics}. 4th edition.
\item John Fox. \emph{Applied Regression Analysis and Generalized Linear Models}. 2nd edition.
%\item \href{http://cran.r-project.org/doc/contrib/Owen-TheRGuide.pdf}{Owen. \emph{The R Guide}. At: http://cran.r-project.org/doc/contrib/Owen-TheRGuide.pdf}
%\item \href{http://cran.r-project.org/doc/manuals/R-intro.pdf}{Venables and Smith. \emph{An Introduction to R}. At: http://cran.r-project.org/doc/manuals/R-intro.pdf}
%\item \href{http://cran.r-project.org/doc/contrib/Verzani-SimpleR.pdf}{Verzani. \emph{Simple R}. At: http://cran.r-project.org/doc/contrib/Verzani-SimpleR.pdf}
%\item Freedman, David, Robert Pisani, and Roger Purves. \emph{Statistics}. 4rd Edition. New York: Norton. (statistics basics)
\item Andrew, Gelman and Jennifer Hill. \emph{Data Analysis Using Regression and Multilevel/Hierarchical Models}. Cambridge University Press. (regression modeling)
\item Imai, Kosuke. Quantitative Social Science. An Introduction.  \emph{Princeton University Press}
%\item Gill, Jeff. \emph{Essential Mathematics for Political and Social Research}. 1st Edition. 2nd printing. New York: Cambridge University Press.
\item[] Math and Probability
\item Simon, Carl and Blume, Lawrence. \emph{Mathematics for Economists}. New York: Norton.
\item  Bertsekas, Dimitri P. and Tsitsiklis, John (BT) Introduction to Probability Theory (second edition)

%\item [] Visualization:
%\item Cleveland, William S. \emph{Visualizing Data}. Summit, NJ: Hobart Press.
%\item Tufte, Edward. \emph{The Visual Display of Quantitative Information, 2nd Edition}. Cheshire, CN: Graphics Press.

\item[] Computing:
\item Paul Teetor. 2011. \emph{R Cookbook}. O'Reilly Media.
\item Fox, John and Sanford Weisberg. \emph{An R Companion to Applied Regression}. 2nd ed. (R, with focus on regression modeling)
\item Murrell, Paul. \emph{R Graphics}. Chapman \& Hall.
\item Wickham, Hadley. \emph{ggplot2: Elegant Graphics for Data Analysis}. Springer.
\item Sarkar, Deepayan. \emph{Lattice: Multivariate Data Visualization with R}. Springer.
\end{itemize}

%\item Agresti, Alan and Finlay, Barbara. 1997. \emph{Statistical Methods for the Social Sciences} Upper Saddle River, NJ: Prentice Hall.

% \item Weisberg, Sanford. 2005. \emph{Applied Linear Regression}. 3rd Edition. Hoboken, NJ: John Wiley.

%\item Kennedy, Peter. 2003. \emph{A Guide to Econometrics}. 5th Edition. Malden. Blackwell.
%\item Wonnacott, Thomas H. and Ronald J. Wonnacott. 1990. \emph{Introductory Statistics}. 5th Edition. New York: Wiley.
  %\item Pindyck, Robert S. and Daniel L. Rubinfeld. 1998. \emph{Econometric
 % Models and Economic Forecasts}. 4th Edition. Boston: McGraw Hill.
%  \item Dalgaard, Peter. 2002. \emph{Introductory Statistics with
% R}. New York: Springer.
%  \item Venables, W.N. and B.D. Ripley. 2002. \emph{Modern Applied
%    Statistics with S-PLUS}. New York: Springer
% \item Gonick, Larry and Smith, Woollcott. 1993. \emph{The Cartoon Guide to Statistics} New York: Harper.


%\section{Introduction}
%
%\begin{itemize}
%  \item Overview and Course Requirements
%%  \item Examples of Statistical Inference in Political Science
%  \item Course Outline
%\end{itemize}

%\textit{Required Readings:}
%\begin{itemize}
%  \item Fox97, Chapter 1
%  \item Fox02, Preface, Sections 1.1 and 1.3
%  \item Cleveland, Preface, Chapter 1
%  \item Achen, Chapter 1
%\end{itemize}
%
%\textit{Optional Readings:}
%\begin{itemize}
%\item Agresti and Finlay, Chapters 1 and 2
%  \item Freedman, et al., Chapters 1 and 2
%  \item Barnett, Vic. 1999. \emph{Comparative Statistical Inference}. New York: Wiley.
%  \item Achen, Christopher and Larry Bartels. 2004. ``Musical Chairs: Pocketbook Voting and the Limits of Democratic Accountability.'' Paper presented at the 2004 Annual Meeting of the American Political Science Association (available on course website).
%\end{itemize}

%\section{Elementary Probability Theory}
%\begin{itemize}
%  \item Why Do We Need Probability?
%  \item Probability Axioms
%  \item Marginal, Joint and Conditional Probability
%  \item Law of Total Probability
%  \item Bayes' Rule
%  \item Independence
%\end{itemize}
%\textit{Required Readings:}
%\begin{itemize}
%  \item Bertsekas and Tsitsiklis, Chapter 1
%  \item Wooldridge, Appendix A
%\end{itemize}
%
%\textit{Required Readings:} \begin{itemize}
% \item Gill, Chapters 7, 8
%  \item Fox97 Appendix D.1-D.4%,  D.5-D.5.2
%\end{itemize}
%[I might include the elementary probability chapters from Jeff Gill's book as well]
%\textit{Optional Readings:} \begin{itemize}
%  \item Gonick and Smith, Chapters 2-5
%  \item Wooldridge Appendix B
%  \item Freedman et al., Chapters 13-15
% \item Wonnacott \& Wonnacott, Sections 3.1-3.6, 4.1-4.5, 5.1-5.4, 6.1-6.3
% \item Diaconis, Persi and Mosteller, Frederick. 1989. ``Methods for Studying Coincidences.'' \emph{JASA} 84: 853-861.
%  \item Gelman, Andrew; Gary King; and John
 %   Boscardin. 1998. ``Estimating the Probability of Events that Have
 %   Never Ocurred: When is Your Vote Decisive?.'' \emph{JASA}. 93:
 %   1-9. (available via JSTOR)
%\end{itemize}

%\section{Random Variables and Probability Distributions}
%\begin{itemize}
%  \item Discrete and Continuous Random Variables
%  \item Measures of Location
%  \item Measures of Dispersion
%  \item Probability Distributions
%\end{itemize}
%\textit{Required Readings:}
%\begin{itemize}
%  \item Bertsekas and Tsitsiklis, Chapters 2.1--2.4 \& 3.1--3.3
%  \item Wooldridge, Appendix B.1 \& B.3
%\end{itemize}
%
%
%\section{Multiple Random Variables}
%\begin{itemize}
%  \item Joint and Conditional Distributions
%  \item Conditional Expectation
%  \item Covariance and Independence
%\end{itemize}
%\textit{Required Readings:}
%\begin{itemize}
%  \item Bertsekas and Tsitsiklis, Chapters 2.5--2.8, 3.4--3.7, 4.2 \& 4.3
%  \item Wooldridge, Appendix B.2 \& B.4--B.5
%\end{itemize}


\section{Univariate Statistical Inference}

\subsection{Point Estimation}
\begin{itemize}
   \item Properties of Estimators
   \item Sampling Distribution
   \item Elementary Asymptotic Theory
\end{itemize}
\textit{Required Readings:}
\begin{itemize}
  \item Wooldridge, Appendix C1-C4
  \item Unless you are well familiar with this material you need to review: Wooldridge, Appendix A \& B
\end{itemize}

\subsection{Interval Estimation}

\begin{itemize}
   \item Confidence Intervals
\end{itemize}
\textit{Required Readings:}
\begin{itemize}
  \item Wooldridge, Appendix C5
\end{itemize}
\subsection{Hypothesis Testing}
\begin{itemize}
   \item Logic of Statistical Testing
   \item p-Values
\end{itemize}
\textit{Required Readings:}
\begin{itemize}
  \item Wooldridge, Appendix C6-C7 
\end{itemize}

% \subsection{Bootstrap Inference}

% \begin{itemize}
%   \item Non-parametric boostrap
%   \item Block bootstrap
%   \end{itemize}
% \textit{Readings:}
% \begin{itemize}
%   \item Bradley Efron and R.J. Tibshirani, An Introduction to the Bootstrap
% (Chapman \& Hall/CRC Monographs on Statistics \& Applied
% Probability, 1993).
% \item A. C. Davison and D. V. Hinkley, Bootstrap Methods and their
% Application (Cambridge Series in Statistical and Probabilistic
% Mathematics, 1997).
% \end{itemize}


%\textit{Required Readings:} \begin{itemize}
% \item Agresti and Finlay, Sections 4.3-5.3, 6.1 - 6.5
% \item Wooldridge, Appendix C
%  \item Fox97 Appendix D.5-D.5.2
%  \item American National Election Studies website (http://www.electionstudies.org/)
%\end{itemize}

%\textit{Optional Readings:} \begin{itemize}
%  \item Gonick and Smith, Chapters 6-8
%  \item Freedman et al., Chapters 16-26
%\end{itemize}


\section{What is Regression?}

\begin{itemize}
\item Nonparametric Regression
\item Linear Regression
\item Bias-Variance Tradeoff
\end{itemize}
\textit{Required Readings:} \begin{itemize}
  \item Wooldridge, Chapter 1
\end{itemize}

%\textit{Required Readings:} \begin{itemize}
%  \item Fox97, Sections 2-3.2
%  \item Fox02, Sections 2-3.2
% \item Tatem, Andrew J; Carlos A. Guerra; Peter M. Atkinson; and
%   Simon I. Hay. 2004. ``Momentous Sprint at the 2156 Olympics.''
%    \emph{Nature} 431 (30 September): 525. (available at course website)
%\end{itemize}

%\textit{Optional Readings:} \begin{itemize}
% \item Cleveland, Chapter 2 pp. 16 - 33 and Chapter 3 pp. 86-101
%  \item Achen, Chapter 2
%   \item Tufte, Edward. 2001. \emph{The Visual Display of Quantitative Information, 2nd Edition}. Cheshire, CN: Graphics Press.
%  \item Cleveland, William S. and Robert McGill. 1987. ``Graphical
 %Perception: The Visual Decoding of Quantitative Information on
 % Graphical Displays of Data.'' (with discussion) \emph{JRSS A}. 150:
%  192-229.  (available via JSTOR)
%  \item Cleveland, William S. and Robert McGill. 1984. ``Graphical
%  Perception: Theory, Experimentation, and Application to the
%  Development of Graphical Methods.'' \emph{JASA}. 79: 531-554
%  (available via JSTOR)
 % \item Cleveland, William S.; Persi Diaconis; and Robert
%  McGill. 1982. ``Variables on Scatterplots Look More Highly
%  Correlated When the Scales are Increased.'' \emph{Science}. 216:
%  1138-1141 (available via JSTOR)
% \end{itemize}


%\section{Visualization}

%\textit{Required Readings:} \begin{itemize}
 % \item Cleveland, Chapters 4, 5, and 6 (review chapters 2 and 3)
%\end{itemize}
%\begin{Background Reading} % \item Freedman et al., Chapters 3-7
%\item Cleveland, William S. 1994. \emph{The Elements of Graphing
 % Data}. Summit, NJ: Hobart Press.
%\end{Background Reading}


%\textit{Optional Readings:} \begin{itemize}
  %\item Cleveland, William S. and Robert McGill. 1987. ``Graphical
 % Perception: The Visual Decoding of Quantitative Information on
  %Graphical Displays of Data.'' (with discussion) \emph{JRSS A}. 150:
  %192-229.  (available via JSTOR)
  %\item Cleveland, William S. and Robert McGill. 1984. ``Graphical
  %Perception: Theory, Experimentation, and Application to the
  %Development of Graphical Methods.'' \emph{JASA}. 79: 531-554
  %(available via JSTOR)
  %\item Cleveland, William S.; Persi Diaconis; and Robert
  %McGill. 1982. ``Variables on Scatterplots Look More Highly
  %Correlated When the Scales are Increased.'' \emph{Science}. 216:
  %1138-1141 (available via JSTOR)
  %\item Trellis Display homepage at Bell Labs\\ \texttt{http://cm.bell-labs.com/cm/ms/departments/sia/project/trellis/}
%\end{itemize}



\section{Simple Linear Regression}

\begin{itemize}
\item Mechanics of Ordinary Least Squares
\item Linear Model Assumptions
\item Properties of the Least Squares Estimator
\item Gauss-Markov Theorem
\item Testing and Confidence Intervals
\item Large Sample Inference
\end{itemize}
\textit{Required Readings:} \begin{itemize}
  \item Wooldridge, Chapter 2
  \item  Hill, Griffiths, and Lim, Chapters 2 \& 3 
\end{itemize}
%\textit{Optional Readings:} \begin{itemize}
%\item Tatem, Andrew J; Carlos A. Guerra; Peter M. Atkinson; and Simon I. Hay. 2004. ``Momentous Sprint at the 2156 Olympics.'' \emph{Nature} 431 (30 September): 525.
%\end{itemize}


%\textit{Required Readings:} \begin{itemize}
%  \item Fox97, Sections 5.1 and 6.1
%  \item Fox02, Section 4.1.1
%  \item Wooldridge, Chapter 2
%  \item Fox97, Sections 4-4.4
%  \item Fox02, Section 3.4

%\item Osberg, Lars and Timothy Smeeding. 2004. `` `Fair' Inequality? An International Comparison of Attitudes to Pay Differentials'' \emph{American Sociological Review}, 71: 450-473.
 %  \item Epstein, Lee and Carol Mershon. 1996. ``Measuring Political
   % Preferences.'' \emph{American Journal of Political Science}, 40:
   % 261-294. (available via JSTOR)
%\end{itemize}
%\textit{Optional Readings:} \begin{itemize}
%  \item Freedman et al., Chapters 8-12
%\item Wonnacott \& Wonnacott, Chapters 11 and 12
%  \item Tufte, Edward. 1974. \emph{Data Analysis for Politics and
%    Policy}. Englewood Cliffs, NJ: Prentice-Hall. Chapter 3.
%    \item Stigler, Steven M. 1986. \emph{The History of Statistics: The
%  Measurement of Uncertainty before 1900}. Cambridge, MA: Harvard
%  University Press. Chapters 1,8
%\end{itemize}

\section{Linear Regression with Two Regressors}
\subsection{Mechanics of Regression with Two Regressors}
\begin{itemize}
\item Motivation for Multiple Regression
\item Mechanics for OLS with Two Regressors
\end{itemize}
\textit{Required Readings:} \begin{itemize}
  \item Wooldridge, Chapter 3
\end{itemize}
\begin{itemize}
\item Inference for OLS with Two Regressors
\end{itemize}
\textit{Required Readings:} \begin{itemize}
  \item Wooldridge, Chapter 4 \& 5
\end{itemize}

\subsection{Omitted Variables and Multicollinearity}
\begin{itemize}
\item Omitted Variable Bias
\item Multicollinearity
\end{itemize}
\textit{Required Readings:} \begin{itemize}
  \item Wooldridge, Chapters 6
      \item Alternative: Hill, Griffiths, and Lim, Chapter 6 (course website)
\end{itemize}
\subsection{Dummy Variables, Interactions and Polynomials}
\begin{itemize}
\item Dummy Variables
\item Interaction Terms
\item Polynomials and Logarithms
\end{itemize}
\textit{Required Readings:} \begin{itemize}
  \item Wooldridge, Chapter 7
    \item Hill, Griffiths, and Lim, Chapter 4 \& 7
\end{itemize}

%\textit{Optional Readings:} \begin{itemize}
%\item Brambor, Thomas,  William Clark, and Matt Golder. 2005. ``Understanding Interaction Models: Improving Empirical Analyses.'' \emph{Political Analysis}. 13: 1-20.
%\item Braumoeller, Bear. 2004. ``Hypothesis Testing and Multiplicative Interaction Terms.'' \emph{International Organization}. 58: 807-820.
%\end{itemize}


\section{Multiple Linear Regression}
\subsection{Mechanics of Multiple Regression}
\begin{itemize}
\item Review of Matrix Algebra and Vector Calculus
\item Mechanics of Multiple Linear Regression
\end{itemize}
\subsection{Statistical Inference with Multiple Regression}
\begin{itemize}
\item Statistical Inference for Multiple Linear Regression
\item Testing Multiple Hypotheses
\end{itemize}
\textit{Required Readings:} \begin{itemize}
  \item Wooldridge, Appendix D \& E
\end{itemize}

%\textit{Required Readings:} \begin{itemize}
%\item Simon and Blume. 1994. \emph{Mathematics for
%  Economists}. Sections 8.1-8.4 (handout)
% \item Gill Chapters 3, 4, 5, 6

% \item Gill. 2006. Chapter 3.1, 3.2 pp. 82-88 (81-86 on web)
 %\item Gill. 2006. Chapter 3.3, 3.4 pp. 100-112 (102-110 on web)
 %\item Gill. 2006. Chapter 3.5
 %\item Gill. 2006. Chapter 4.3 pp. 139-142 (132-134 on web)
 %\item Gill. 2006. Chapter 4.6
 %\item Fox97, Appendix B
%\item Fox97 Section 10.1,10.2
% \item Fox97, Appendices C.1, C.3
%  \item Fox97, Sections 9.1 pp. 204-206, 9.2
%  \item Fox02, Sections 1.2 and 4.1

  %\item Tsebelis, George. 1999. ``Veto Players and Law Production in
   % Parliamentary Democracies: An Empirical Analysis." \emph{The
     % American Political Science Review}, 93: 591-608. (available via JSTOR)
%\end{itemize}

%\section{Properties of the Least Squares Estimator, Causality, and Measurement Error - Nov. 19}

%\textit{Required Readings:} \begin{itemize}

%\end{itemize}
%\textit{Optional Readings:} \begin{itemize}
%  \item Freedman et al., Chapters 26-29
%  \item Achen, Chapter 3
%  \item Simon and Blume. 1994. \emph{Mathematics for
%  Economists}. Sections 8.1-8.4
% \item Cleveland. Chapter 4
%\end{itemize}

%\section{Causal Inference for Least Squares Regression}

%\textit{Required Readings:} \begin{itemize}
%\item Pearl, Judea \emph{Causality} (Handout)
%\end{itemize}


%\textit{Required Readings:} \begin{itemize}
%  \item Fox97, Sections 6.2-6.5, 9.3-9.7
%  \item Fox02, Sections 4.5 and 4.6
%  \item Braumoeller, Bear. 2004. ``Hypothesis Testing and
%  Multiplicative Interaction Terms.'' \emph{International
%  Organization}. 58: 807-820.
%\end{itemize}
%\textit{Optional Readings:} \begin{itemize}
%\item Wooldridge, Chapters 3,4,5
% \item Kennedy, Appendix A, 418-422
%  \item Freedman et al., Chapters 26-29
%   \item Achen, Chapter 4
%\end{itemize}

%\section{Causal Inference for Least Squares Regression}

%\textit{Required Readings:} \begin{itemize}
%\item Pearl, Judea \emph{Causality} (Handout)
%\end{itemize}

%\section{Dummy Variables, Interaction Terms, and Polynomial Regression - Dec. 17}

%\textit{Required Readings:} \begin{itemize}
 % \item Fox97, Chapter 7
   %\item Fox97, Section 14.2-14.2.1
  %\item Fox02, Section 4.2
  %\item Brambor, Thomas,  William Clark, and Matt
  %Golder. 2005. ``Understanding Interaction Models: Improving
  %Empirical Analyses.'' \emph{Political Analysis}. 13: 1-20.
  %\item Braumoeller, Bear. 2004. ``Hypothesis Testing and
  %Multiplicative Interaction Terms.'' \emph{International
  %Organization}. 58: 807-820.

 % \item Gelman, Andrew and Gary King. 1990. ``Estimating the
  %Incumbency Advantage without Bias.'' \emph{AJPS}. 34:
  %1142-1164. (available via JSTOR)
  %\item Wilkerson, John D. 1999. `` `Killer' Amendments in Congress."
    %\emph{The American Political Science Review}, Vol. 93:
    %535-552. (available via JSTOR)
%\end{itemize}


\section{Diagnosing and Fixing Problems in Linear Regression}
\subsection{Outliers and Influential Observations}
	\begin{itemize}
	\item Plotting Residuals
	\item Standardized and Studentized Residuals
	\item Added Variable and Component Residual Plots
	\item Leverage and Influence
\end{itemize}
\textit{Required Readings:} 
	\begin{itemize}
  \item Fox, Chapter 11 
\end{itemize}
	
\subsection{Heteroskedasticity, Serial Correlation and Clustering}
	\begin{itemize}
	\item Weighted Least Squares
	%\item Generalized Least Squares
	\item Heteroskedasticity-robust Standard Errors
	\item Cluster-robust Standard Errors
	\item Autocorrelation
\end{itemize}
%	\end{itemize}
%\subsection{Measurement Error}
%	\begin{itemize}
%	\item Types of Measurement Errors
%	\item Measurement Error in the Dependent Variable
%	\item Measurement Error in an Independent Variable
%	\end{itemize}
\textit{Required Readings:} \begin{itemize}
  \item Wooldridge, Chapters 8--9
\item  Fox, Chapter 12
  \item Hill, Griffiths, and Lim, Chapter 8 
  \end{itemize}
%\textit{Optional Readings:} \begin{itemize}
% \item Jackman, Robert W. 1987.``The Politics of Economic Growth in
%     the Industrial Democracies, 1974-80: Leftist Strength or North Sea
%     Oil?'' \emph{The Journal of Politics}, Vol. 49, No. 1,
%     pp. 242-256. (available via JSTOR)
%        \item Wand, Jonathan; Kenneth Shotts; Jasjeet Sekhon; Walter Mebane;
%    Michael Herron; and Henry Brady. 2001 ``The Butterfly Did It: The
%    Aberrant Vote for Buchanan in Palm Beach County, Florida.''
%    \emph{APSR}. 95: 793-810.
%\end{itemize}


%\textit{Required Readings:} \begin{itemize}
%  \item Fox97, 12.2, 14.1
%  \item Fox02, Sections 6.3
%  \item Fox02, Appendix: ``Time-Series Regression and Generalized Least Squares''
%    \item Beck, Nathaniel and Katz, Jonathan. 1995. ``What to do (and not to do) with Time Series Cross-Sectional Data.'' \emph{APSR}. 89:
%  634-647. (available via JSTOR)
%\end{itemize}
%\textit{Optional Readings:} \begin{itemize}
%\item Berk, Kenneth. 1998. ``Regression Diagnostic Plots in 3-D'' \emph{Technometrics}. 40: 39-47.
%   \item Weisberg, Sanford. 2005. \emph{Applied Linear Regression}. 3rd Edition. Hoboken, NJ: John Wiley.
% \item Achen, Chapters 5, 6, and 7
%\item Wooldridge, Chapters 10-12
%\end{itemize}


%\textit{Required Readings:} \begin{itemize}
%  \item Fox97, Chapter 11
%  \item Fox02, Section 6.1
%
%    \item Fox97, 12.1,12.3,12.4, 12.7
%  \item Fox02, Sections 6.2, 6.3, and 6.4
%  (available via JSTOR)
 % \item Lange, Peter and Geoffrey Garrett. 1985. ``The Politics of
  %  Growth: Strategic Interaction and Economic Performance in the
   % Advanced Industrial Democracies, 1974-1980.'' \emph{The Journal of
   % Politics}. Vol. 47. No. 3. pp. 792-827. (available via JSTOR)

%  \item R. Michael Alvarez, Geoffrey Garrett, and Peter
  % Lange. 1991. ``Government Partisanship, Labor Organization, and
   % Macroeconomic Performance.''  \emph{American Political Science
    %  Review}, Vol 85, No. 2, pp. 539-556. (available via JSTOR)
%  \item Beck, Nathaniel; Jonathan Katz; R. Michael Alvarez; Geoffrey
 %  Garrett; and Peter Lange. 1993. ``Government Partisanship, Labor
    %Organization, and Macroeconomic Performance: A Corrigendum.''
   % \emph{American Political Science Review} Vol 87 No. 4,
   % pp. 945-948. (available via JSTOR)
%\end{itemize}

%\textit{Optional Readings:} \begin{itemize}
%  \item Weisberg, Sanford. 2005. \emph{Applied Linear Regression}. 3rd Edition. Hoboken, NJ: John Wiley.
%   \item Hastie, Trevor; Robert Tibshirani; and Jerome
%  Friedman. 2001. \emph{The Elements of Statistical Learning}. New
%  York: Springer.
%  \item Hastie, Trevor and Rob Tibshirani. 1990. \emph{Generalized
%    Additive Models}. London: Chapman \& Hall.

%\end{itemize}
%
%\section{Extensions and Advanced Topics (time permitting)}
%\begin{itemize}
%\item Nonlinear Regression Models
%	\begin{itemize}
%	\item Logit and Probit Models
%	\item Generalized Linear Models
%	\end{itemize}
%\item Semiparametric and Nonparametric Regression Models
%	\begin{itemize}
%	\item Generalized Additive Models
%	\end{itemize}
%\end{itemize}
%\textit{Required Readings:} \begin{itemize}
%  \item Wooldridge, Chapter 17.1
%\end{itemize}
%\textit{Optional Readings:}
%\begin{itemize}
% \item Beck, Nathaniel and Simon Jackman. 1998. ``Beyond Linearity by
%  Default: Generalized Additive Models.'' \emph{AJPS}. 42:
%  596-627.
%\end{itemize}

%\textit{Required Readings:} \begin{itemize}
%  \item Fox97, Chapter 14.2-14.4
%  %   \item Fox97, Sections 16.1-16.1.3
%   \item Fox97, Sections 15.1-15.1.4
%   \item Fox97, Sections 5.1, 5.2
%  \item Fox02, Online Appendices
 % \item Venables and Ripley. 2002. pp. 156-163 (handout)
 % (available via JSTOR)
  %\item Western, Bruce. 1995. ``Concepts and Suggestions for Robust
  %Regression Analysis.'' \emph{AJPS}. 39: 786-817. (available via JSTOR)

%  \item Quinn, Kevin ``Visualizing Multivariate Outliers and Leverage
%    Points.'' (handout)
%\end{itemize}
%\textit{Optional Readings:} \begin{itemize}
%
%  \item Hampel, Frank. 1974. ``The Influence Curve and Its Role in
%  Robust Estimation.'' \emph{JASA}. 69: 383-393.
%  \item Hampel, F.R., E.M. Ronchetti, P.J. Rousseeuw, and W.A. Stahel,
%  1986. \emph{Robust Statistics: The Approach Based on Influence
%  Functions.} New York: Wiley.
%  %\item Huber, P. J. 1964. ``Robust Estimation of a Location
%  Parameter.'' \emph{Annals of Mathematical Statistics}. 35: 73-101.
%  \item Rousseeuw, R.J. and A.M. Leroy. 1987. \emph{Robust Regression
%  and Outlier Detection}. New York: Wiley.
%  \item Stefanski, L.A. 1991. ``A Note on High-Breakdown Estimators.''
%  \emph{Statistics and Probability Letters}. 11: 353-358.
%\end{itemize}


%\textit{Optional Readings:} \begin{itemize}
%\item Rousseeuw, Peter J. and Bert C. van Zomeren. 1990. ``Unmasking Multivariate Outliers and Leverage Points (with Discussion).''\emph{JASA}. 85: 633-651.
%\item Hampel, Frank. 2001. ``Robust Statistics: A Brief Introduction
%  and Overview.'' \href{http://e-collection.ethbib.ethz.ch/ecol-pool/incoll/incoll\_105.pdf}{Link}.
%\end{itemize}


%\section{Nonparametric Regression and Additive Models Revisited, Bootstrapping, and an Introduction to Logit and Probit - Dec. 17}

%\textit{Required Readings:} \begin{itemize}
 % \item Fox97, Sections 14.4
%  \item Fox97, Sections 16.1-16.1.3
  %\item Fox97, Sections 15.1-15.1.4
%\item Fox97, Sections 5.1, 5.2
 % \item Stock and Watson, 2003. Chapter 6 (handout)
%  \item Friedrich, Robert. 1982. ``In Defense of Multiplicative Terms
 % in Multiple Regression Equations.'' \emph{American Journal of
 % Political Science}. 26: 797-833. (available via JSTOR)
  % \item Neto, Octavio Amorim, and Gary W. Cox. 1997. ``Electoral
   % Institutions, Cleavage Structures, and the Number of Parties."
   % \emph{American Journal of Political Science}, 41:
   % 149-174. (available via JSTOR)
  %\item Kousser, T. ``Retrospective Voting and Strategic Behavior in European
 %   Parliament Elections.''\emph{Electoral Studies},
   % Vol. 23, (2004), pp. 1-21.
 % \item Venables and Ripley. 2002. Chapter 8. (handout)
%\end{itemize}


%\section{Introduction to Logit and Probit Models}

%\textit{Required Readings:} \begin{itemize}
%\item Fox97, Chapter 15.1-15.1.4
%\item Fox97, Sections 5.1, 5.2
%\end{itemize}



%King, Gary. 1986. "How Not to Lie With Statistics: Avoiding Common Mistakes in Quantitative Political Science." American Journal of Political Science 30(3):666-87.

%King, Gary. 1991. "Truth is Stranger than Predication, More Questionable than Causal Inference." American Journal of Political Science 35(4):1047-53.

\clearpage


\section*{Causal Inference}
\setcounter{section}{0}

\subsection*{Books}
\begin{itemize}
  \item Angrist, Joshua D. and J\"orn-Steffen Pischke. 2009. \emph{Mostly Harmless Econometrics: An Empiricist's Companion}. Princeton University Press. (A standard reference for applied researchers for most topics covered in the first part of the course.)
  \item Morgan, Stephen L. and Christopher Winship. 2015. \emph{Counterfactuals and Causal Inference: Methods and Principles for Social Research}, Second Edition. Cambridge University Press. 
  \item Wooldridge, Jeffrey. 2010. {\it Econometric Analysis of Cross Section and Panel Data}, 2nd ed. MIT Press.
\item Imbens, Guido and Donald B. Rubin. 2015. {\it Causal Inference for Statistics, Social, and Biomedical Sciences: An Introduction}.1st Edition. Cambridge University Press.
\item Gerber, Alan S., and Donald P. Green. 2012. \textit{Field Experiments}. W. W. Norton. 
\item Rosenbaum, Paul R. 2009. \textit{Design of Observational Studies}.  Springer Series in Statistics.
    \item Rosenbaum, Paul R. 2002. \textit{Observational Studies}. Springer-Verlag. 2nd edition.
  \item Pearl, Judea. 2009. Causality: Models, Reasoning, and Inference. New York: Cambridge University Press. 2nd edition. 
        \item Manski, Charles F. 1995. \textit{Identification Problems in the Social Sciences}. Cambridge: Harvard University Press.
  \item Rubin, Donald. 2006. \textit{Matched Sampling for Causal Effects}. Cambridge University Press.
\end{itemize}


%\textbf{Optional Books}
%\begin{itemize}
%\item \emph{The following books are optional but may prove
%  useful for additional coverage of some of the
%  course topics.}
%%\item \emph{Reference Book for Panel Methods}
%%\begin{itemize}
%%  \item Wooldridge, Jeffrey M. 2002. Econometric Analysis of Cross Section and Panel Data. MIT Press.
%%\end{itemize}
%\item  \emph{Causal Inference}
%\begin{itemize}
%\item Rosenbaum, Paul R. 2009. \textit{Design of Observational Studies}.  Springer Series in Statistics.
%    \item Rosenbaum, Paul R. 2002. \textit{Observational Studies}. Springer-Verlag. 2nd edition.
%  \item Pearl, Judea. 2009. Causality: Models, Reasoning, and Inference. New York: Cambridge University Press. 2nd edition.
%        \item Manski, Charles F. 1995. \textit{Identification Problems in the Social Sciences}. Cambridge: Harvard University Press.
%\end{itemize}
%\item  \emph{Matching}
%\begin{itemize}
%  \item Rubin, Donald. 2006. \textit{Matched Sampling for Causal Effects}. Cambridge University Press.
%\end{itemize}
%\item  \emph{Model Based Inference}
%\begin{itemize}
%  \item Cameron, Colin and Pravin Trivedi. 2005. {\it
%        Microeconometrics: Methods and Applications}. Cambridge
%      University Press. (Slightly less standard, but covers most of the topics throughout the
%      course.)
%        \item McCullagh, P. and J. A. Nelder. 1989. {\it Generalized Linear Models}, 2nd ed. Chapman and Hall/CRC. (Generalized linear models)
%  \item Efron, Bradley and Robert J. Tibshirani. 1994. {\it An
%      Introduction to the Bootstrap}. Chapman and Hall/CRC. (Bootstrap)
%  \item Kalbfleisch, John D. and Ross L. Prentice. 2002. {\it The Statistical
%  Analysis of Failure Time Data}, 2nd ed. Wiley. (Survival analysis)
%\end{itemize}
%\end{itemize}

\section{The Potential Outcome Model}

\begin{itemize}
\item Counterfactual Responses and the Fundamental Identification Problem
\item Estimands and Assignment Mechanisms
\item Heterogeneity and Selection
\end{itemize}

\emph{Readings}
($\star$) indicates required
\begin{itemize}
\item Morgan and Winship: Chapter 1-2. ($\star$)
\item Angrist and Pischke: Chapter 1. ($\star$)
\item Holland, Paul W. 1986.  \href{http://www.jstor.org/stable/2289064} {Statistics and Causal Inference}. \textit{Journal of the American Statistical Association} 81(396): 945-960. ($\star$)
\item Sekhon, Jasjeet S. 2004.  \href{http://journals.cambridge.org/abstract_S1537592704040150} {Quality Meets Quantity: Case Studies, Conditional Probability and Counterfactuals}. \textit{Perspectives on Politics} 2 (2): 281-293.
\end{itemize}


\section{Randomized Experiments}

\begin{itemize}
\item Identification of Causal Effects under Randomization
\item Implementation, Estimation, Diagnostics, Blocking
\item Threats to Validity
\end{itemize}

\emph{Readings: Theory of Experiments}
\begin{itemize}
\item  Angrist and Pischke: Chapter 2. ($\star$)
\item Rosenbaum, Paul R. 2002. \textit{ Observational Studies}. Springer-Verlag. 2nd edition. Chapter 2.
\item Gerber, Alan S., and Donald P. Green. 2012. \textit{Field Experiments}. W. W. Norton. Chapters 2-4.
\item Neyman, Jerzy. 1923 [1990]. \href{http://www.jstor.org/stable/2245382}{On the Application of Probability Theory to Agricultural Experiments. Essay on Principles. Section 9}. \textit{Statistical Science} 5 (4): 465-472. Trans. Dorota M. Dabrowska and Terence P. Speed.
\item Lin, Winston. Forthcoming. \href{http://arxiv.org/pdf/1208.2301.pdf}{Agnostic Notes on Regression Adjustments to Experimental Data: Reexamining Freedman's Critique}. \textit{The Annals of Applied Statistics}.
\end{itemize}

\emph{Readings: Application of Experiments}
\begin{itemize}
  \item Olken, Benjamin. 2007. \href{http://www.journals.uchicago.edu/doi/abs/10.1086/517935}{Monitoring corruption : Evidence from a field experiment in Indonesia}. \textit{Journal of Political Economy} 115 (2): 200-249.
  \item Gerber, Alan S., Donald P. Green and Christopher W. Larimer. 2008. \href{http://journals.cambridge.org/action/displayAbstract?fromPage=online&aid=1720748}{Social Pressure and Voter Turnout: Evidence from a Largescale Field Experiment}. \textit{American Political Science Review} 102 (1): 1-48. ($\star$)
\item Wantchekon, Leonard. 2003. \href{http://www.jstor.org/stable/25054228} {Clientelism and Voting Behavior: Evidence from a Field Experiment in Benin} \textit{World Politics} 55 (3), April: 399-422.
  \item Chattopadhyay, Raghabendra and Esther Duflo. 2004.  \href{http://www.jstor.org/stable/3598894} {Women as Policy Makers: Evidence from a Randomized Policy Experiment in India}. \textit{Econometrica}, 72 (5): 1409-1443.
  \end{itemize}

  \emph{Readings: Application of Natural Experiments}
\begin{itemize}
%\item DellaVigna, Stefano, and Ethan Kaplan. 2007.  \href{http://www.mitpressjournals.org/doi/abs/10.1162/qjec.122.3.1187} {The Fox News Effect: Media Bias and Voting}. Quarterly Journal of Economics 122(3): 1187-1234.
\item Hyde, Susan D. 2007.  \href{http://muse.jhu.edu/journals/world_politics/v060/60.1.hyde.html} {The Observer Effect in International Politics: Evidence
from a Natural Experiment}. \textit{World Politics} 60(1): 37-63. ($\star$)
\item Ferraz, Claudio, and Federico Finan. 2008.  \href{http://qje.oxfordjournals.org/content/123/2/703.short%E2%80%8E} {Exposing Corrupt Politicians: The
Effects of Brazil's Publicly Released Audits on Electoral Outcomes}. \textit{Quarterly
Journal of Economics} 123(2): 703-45.
\item Ho, Daniel E., and Kosuke Imai. 2008.  \href{http://poq.oxfordjournals.org/cgi/content/abstract/72/2/216} {Estimating Causal Effects of Ballot
Order from a Randomized Natural Experiment: The California Alphabet Lottery},
1978-2002.\textit{ Public Opinion Quarterly} 72(2): 216-40.
\item Dunning, Thad. 2012. \textit{Natural Experiments in the Social Sciences: A Design-Based Approach}. New York: Cambridge University Press.
\end{itemize}

\emph{Readings: Experiments Review Articles}
\begin{itemize}
\item Palfrey, Thomas. 2009.
  \href{http://arjournals.annualreviews.org/doi/pdf/10.1146/annurev.polisci.12.091007.122139}
  {Laboratory Experiments in Political Economy}.\textit{ Annual Review of
  Political Science} 12: 379-388.
\item Druckman, James N., Donald P. Green, James H. Kuklinski, and
  Arthur Lupia. 2006.
  \href{http://journals.cambridge.org/abstract_S0003055406062514} {The
    Growth and Development of Experimental Research in Political
    Science}.\textit{ American Political Science Review} 100(4): 627-635.
\item Green, Donald P., Peter M. Aronow, and Mary C. McGrath. 2012. \href{http://www.tandfonline.com/doi/abs/10.1080/17457289.2012.728223}{Field Experiments and the Study of Voter Turnout}. \textit{Journal of Elections, Public Opinion \& Parties}: 1-22.
\item Humphreys, Macartan, and Jeremy Weinstein. 2009.
  \href{http://www.columbia.edu/~mh2245/papers1/HW_ARPS09.pdf} {Field
    Experiments and the Political Economy of Development}. \textit{Annual
  Review of Political Science} 12: 367-378.
\item Harrison, Glenn and John A. List. 2004.
  \href{http://people.hbs.edu/nashraf/Harrison_Field%20Experiments_2004.pdf}
  {Field Experiments}. \textit{Journal of Economic Literature}, XLII:
  1013-1059.
\item List, John A., and Steven Levitt. 2006. \href{http://pricetheory.uchicago.edu/levitt/Papers/jep%20revision%20Levitt%20%26%20List.pdf}
  {What Do Laboratory Experiments Tell Us About the Real World?}
  University of Chicago and NBER.
\item Gaines, Brian J., and James H. Kuklinski. 2007.
  \href{http://pan.oxfordjournals.org/cgi/content/abstract/15/1/1}
  {The Logic of the Survey Experiment Reexamined}. \textit{Political Analysis}
  15: 1-20.
\end{itemize}

\emph{Readings: Useful Methodological Guides for Experiments}

\begin{itemize}

\item Duflo, Esther, Abhijit Banerjee, Rachel Glennerster, and Michael Kremer. 2006.  \href{http://economics.mit.edu/files/806}{Using Randomization in Development Economics: A Toolkit}. \textit{Handbook of Development Economics}.

\item Bloom, Howard S. 2008. ``The Core Analytics of Randomized Experiments for Social Research.'' In The SAGE Handbook of Social Research Methods, eds. Pertti Alasuutar, Leonard Bickman, and Julia Brannen. London: SAGE.

\item Bruhn, Miriam, and David McKenzie. 2009. \href{http://pubs.aeaweb.org/doi/pdfplus/10.1257/app.1.4.200}{In Pursuit of Balance: Randomization in Practice in Development Field Experiments}. \textit{American Economic Journal: Applied Economics} 1(4): 200-232.

%\item Gary King, et.al. 2007. \href{http://gking.harvard.edu/gking/files/spd.pdf}{A Politically Robust Experimental Design for Public Policy Evaluation, with Application to the Mexican Universal Health Insurance Program}. Journal of Policy Analysis and Management 26, 3, 479�506.

% \item Stanford Administrative Panels for the Protection of Human Subjects\\ \href{http://humansubjects.stanford.edu/\#start}{http://humansubjects.stanford.edu/\#start}

\end{itemize}

\section{Causal Effects under Selection on Observables}

\subsection{Selection on Observables}
\begin{itemize}
\item Identification under Selection on Observables
\item Subclassification
\end{itemize}

\emph{Readings}
\begin{itemize}
\item Morgan and Winship: Chapters 3-4. ($\star$)
\item Rubin, Donald B. 2008.  \href{http://arxiv.org/pdf/0811.1640}
  {For Objective Causal Inference, Design Trumps
    Analysis}. \textit{Annals of Applied Statistics} 2(3):
  808-840.
\item Rosenbaum, Paul R. 2002. Observational
  Studies. Springer-Verlag. 2nd edition. Chapter 3.
\item Rosenbaum, Paul R. 2005.
  \href{http://stat.wharton.upenn.edu/~rosenbap/heteroReprint.pdf}
  {Heterogeneity and Causality: Unit Heterogeneity and Design
    Sensitivity in Observational Studies}. \textit{The American
    Statistician} 59: 147-152.
\item Acemoglu, Daron. 2005.
  \href{http://economics.mit.edu/files/4468}
  {Constitutions, Politics, and Economics: A Review Essay on Persson
    and Tabellini's The Economic Effects of
    Constitutions}. \textit{Journal of Economic Literature}
  XLIII: 1025-1048.
  % \item Cochran, W. G. 1968.
  %   \href{http://www.jstor.org/stable/2528036} {The Effectiveness of
  %   Adjustment by Subclassification in Re-moving Bias in
  %   Observational Studies}, \textit{Biometrics}, vol. 24: 295-313.
\end{itemize}

\subsection{Matching Methods}
\begin{itemize}
\item Covariate Matching, Balance Checks, Properties of Matching Estimators
\end{itemize}

\emph{Readings: Matching Theory}
\begin{itemize}
\item Morgan and Winship: Chapter 5. ($\star$)
  % \item Morgan and Harding. 2006.
  %   \href{http://smr.sagepub.com/cgi/content/abstract/35/1/3}
  %   {Matching Estimators of Causal Effects: Prospects and Pitfalls
  %   in Theory and Practice}.
\item Imbens, Guido. 2014. \href{http://www.nber.org/papers/w19959}{Matching Methods in Practice: Three Examples}. \textit{NBER Working Paper 19959}. 
\item Sekhon, Jasjeet S. 2009.
  \href{http://arjournals.annualreviews.org/doi/abs/10.1146/annurev.polisci.11.060606.135444}
  {Opiates for the Matches: Matching Methods for Causal
    Inference}.\textit{ Annual Review of Political Science} 12:
  487-508.($\star$)
\item
Ho, Daniel E., Kosuke Imai, Gary King, and Elizabeth A. Stuart. 2007.
  \href{http://pan.oxfordjournals.org/cgi/content/abstract/mpl013v1}
  {Matching as Nonparametric Preprocessing for Reducing Model
    Dependence in Parametric Causal Inference}. \textit{Political
    Analysis} 15: 199-236.
\item Stuart, Elizabeth A. 2009.
  \href{http://www.biostat.jhsph.edu/~estuart/Stuart-MatchingMethods-StatSci-Dec09.pdf}
  {Matching methods for causal inference: A review and a look forward}
\item Rubin: Chapters 3 to 5.
\item Rosenbaum, Paul R., 1995. \textit{Observational Studies}. New
  York: Springer-Verlag. Chapter 3.
\item Abadie, Alberto and Guido W. Imbens. 2006.
  \href{http://www.jstor.org/stable/3598929} {Large Sample Properties
    of Matching Estimators for Average Treatment Effects},
  \textit{Econometrica} 74: 235-267.
\item Abadie, Alberto, and Guido W.
  Imbens. 2011. \href{http://www.hks.harvard.edu/fs/aabadie/bcmp.pdf}{Bias-Corrected Matching Estimators
    for Average Treatment Effects.} \textit{Journal of Business \&
    Economic Statistics} 29(1): 1-11.
\end{itemize}

\emph{Readings: Matching Applications}
\begin{itemize}
\item Lyall, Jason. 2010.
  \href{http://journals.cambridge.org/action/displayFulltext?type=1&pdftype=1&fid=7449380&jid=PSR&volumeId=104&issueId=01&aid=7449372}
  {Are Co-Ethnics More Effective Counter-Insurgents? Evidence from the
    Second Chechen War}. \textit{American Political Science Review},
  104:1 (February 2010): 1-20.
\item Gordon, Sanford and Gregory Huber. 2007.
  \href{http://www.nowpublishers.com/article/Details/QJPS-6035} {The
    Effect of Electoral Competitiveness on Incumbent
    Behavior}. \textit{Quarterly Journal of Political Science} 2(2): 107-138.
  % \item Sekhon, J. 2004.
  %   \href{http://sekhon.polisci.berkeley.edu/papers/SekhonOpticalMatch.pdf}
  %   {The 2004 Florida Optical Voting Machine Controversy: A Causal
  %   Analysis Using Matching}. Manuscript. UC Berkeley.
\item Eggers, Andrew and Jens Hainmueller. 2009.
  \href{https://web.stanford.edu/~jhain/Paper/APSR2009.pdf} {MPs for Sale?
    Estimating Returns to Office in Post-War British
    Politics}. \textit{American Political Science Review}. 103 (4): 513-533.
\item Gilligan, Michael J. and Ernest J. Sergenti. 2008.
  \href{http://nowpublishers.com/article/Details/QJPS-7051}
  {Do UN Interventions Cause Peace? Using Matching to Improve Causal
    Inference}. \textit{Quarterly Journal of Political Science} 3 (2): 89-122.
\item Sekhon, J., and R. Titiunik. 2012. \href{http://www-personal.umich.edu/~titiunik/papers/SekhonTitiunik2012_APSR.pdf}{When Natural
  Experiments Are Neither Natural nor Experiments}. \textit{American
    Political Science Review} 106(1): 35-57.
\item Sen, Maya. 2014. \href{http://scholar.harvard.edu/files/msen/files/sen_ratings.pdf}{How Judicial Qualification Ratings May Disadvantage Minority and Female Candidates}. \textit{Journal of Law and Courts}. 2 (1): 33-65.

  % \item Simmons, B. and Hopkins, D. 2005.
  %   \href{http://journals.cambridge.org/abstract_S0003055405051920}
  %   {The Constraining Power of International Treaties: Theory and
  %   Methods}. American Political Science Review 99(4): 623-631.
\end{itemize}

\subsection{Propensity Score Methods}

\begin{itemize}
\item Identification, Propensity Score Estimation, Matching on the Propensity Score, Weighting on the Propensity Score, Reweighting methods
\end{itemize}

\emph{Readings: Propensity Score Methods Theory}
\begin{itemize}
\item Morgan and Winship: Chapter 5. ($\star$)
\item Rubin: Chapters 10, 11 and 14 (all with Paul
  R. Rosenbaum).
\item Imbens, Guido W. 2004.
  \href{http://www.mitpressjournals.org/doi/abs/10.1162/003465304323023651}{Nonparametric
    Estimation of Average Treatment Effects under Exogeneity: A
    Review}. \textit{Review of Economics and Statistics} 86 (1): 4-29.
\item Hainmueller,
  Jens. 2012. \href{http://pan.oxfordjournals.org/content/20/1/25.abstract}{Entropy
    Balancing for Causal Effects: A Multivariate Reweighting Method to
    Produce Balanced Samples in Observational
    Studies}. \textit{Political Analysis} 20 (1): 25-46.
\item Glynn, Adam, and Kevin Quinn. 2010. \href{http://pan.oxfordjournals.org/content/18/1/36}{An Introduction to the Augmented Inverse Propensity Weighted Estimator}. \textit{Political Analysis} 18(1): 36-56.
\end{itemize}

\emph{Readings: Propensity Score Methods Applications}
\begin{itemize}
\item Rubin, Donald B. 2001.  \href{http://www.springerlink.com/index/R445GG1778314228.pdf}{Using Propensity Scores to Help Design Observational Studies: Application
to the Tobacco Litigation}. \textit{Health Services and Outcomes Research Methodology} 2 (3-4): 169-188.
\item Blattman, Christopher. 2009. \href{http://www.journals.cambridge.org/abstract_S0003055409090212}{From Violence to Voting: War and Political Participation in Uganda.} \textit{American Political Science Review} 103 (2): 231-247.
\end{itemize}
\subsection{Regression}

\begin{itemize}
\item Agnostic Regression framework, Non-parametric Regression, Identification with Regression
\end{itemize}

\emph{Readings}
\begin{itemize}
\item  Angrist and Pischke: Chapter 3. ($\star$)
\item Morgan and Winship: Chapters 6-7. ($\star$)
\item H\"ardle, W and Linton, O. 1994.  \href{http://web.uconn.edu/tripathi/397/Applied%20nonparametric%20methods.pdf} {Applied Nonparametric Methods}, in R. F. Engle and D. L. McFadden eds. \textit{Handbook of Econometrics}, vol. 4. New York: Elsevier Science.
\item White, H. 1980.  \href{http://www.jstor.org/stable/2526245} {Using Least Squares to Approximate Unknown Regression Functions}. \textit{International Economic Review} 21: 149-170.
\item Hainmueller, J. and Hazlett, C. 2014. \href{http://pan.oxfordjournals.org/content/early/2013/10/10/pan.mpt019}{Kernel Regularized Least Squares: Reducing Misspecification Bias with a Flexible and Interpretable Machine Learning Approach}. \textit{Political Analysis} 22(2): 143-168.
2014.
\end{itemize}

\subsection{Conclusion: Selection on Observables}

\begin{itemize}
\item Can Non-Experimental Method Recover Causal Effects?
\end{itemize}

\emph{Readings: Comparison of Experimental and Non-experimental Methods}
\begin{itemize}
\item Dehejia, Rajeev H. and Sadek Wahba. 1999.  \href{http://www.jstor.org/stable/2669919} {Causal Effects in Non-Experimental Studies: Re-Evaluating the Evaluation of Training Programs}, \textit{Journal of the American Statistical Association} 94 (448): 1053-1062.
\item Heckman, James J., Hidehiko Ichimura and Petra Todd. 1998.  \href{http://www.jstor.org/stable/2566973} {Matching as an Econometric
Evaluation Estimator}, \textit{Review
of Economic Studies} 65: 261-294.
\item Shadish, William R., M.H. Clark, and Peter M. Steiner. 2008.  \href{http://stat-athens.aueb.gr/~jpan/Shadish-JASA2008(1334-1356)-17mr09.pdf} {Can Nonrandomized Experiments Yield Accurate Answers? A Randomized Experiment Comparing Random and Nonrandom Assignments}. \textit{Journal of the American Statistical Association} 103 (484): 1334-1344. ($\star$)
\item Arceneaux, Kevin, Alan S. Gerber, and Donald P. Green. 2006.  \href{http://pan.oxfordjournals.org/cgi/content/abstract/14/1/37} {Comparing Experimental and Matching Methods using a Large-Scale Voter Mobilization Experiment}. \textit{Political Analysis} 14 (1): 1-36.
\item John Concato, Nirav Shah, and Ralph Horwitz. 2000.  \href{http://www.ncbi.nlm.nih.gov/pmc/articles/PMC1557642/} {Randomized, Controlled Trials, Observational Studies, and the Hierarchy of Research Designs}.
\textit{New England Journal of Medicine} 342 (25): 1887-92.
\item Benson, Kjell and Arthur J. Hartz. 2000.  \href{http://www.nejm.org/doi/full/10.1056/NEJM200006223422506} {A Comparison of Observational Studies and Randomized, Controlled Trials}. \textit{New England Journal of Medicine} 342(25): 1878-86.
\end{itemize}




\subsection{Sensitivity Analysis}

\begin{itemize}
\item Nonparametric Bounds
\item Formal sensitivity tests
\end{itemize}

\emph{Readings}
\begin{itemize}
\item Guido W. Imbens. 2003. \href{http://www.jstor.org/stable/3132212} {Sensitivity to Exogeneity Assumptions in Program Evaluation}. \textit{The American Economic Review} 93 (2): 126--32.
\item Morgan and Winship: Chapter 12 ($\star$)
\item Rosenbaum, Paul R. 2002. Observational Studies. Springer-Verlag. 2nd edition. Chapter 4.
\item Manski, Charles F. 1995. \textit{Identification Problems in the Social Sciences}. Cambridge: Harvard University Press. Chapter 2.
%\item Joseph Altonji, Todd E. Elder, and Christopher Taber. 2005.  \href{http://aida.wss.yale.edu/seminars/ecm/ecm04/altonji-040428.pdf} {Selection on Observed and Unobserved Variables: Assessing the Effectiveness of Catholic Schools}. \textit{Journal of Political Economy} Vol. 113: 151-184.
\item VanderWeele, Tyler J. , and Onyebuchi A. Arah. 2011. \href{http://journals.lww.com/epidem/Abstract/2011/01000/Bias_Formulas_for_Sensitivity_Analysis_of.8.aspx}{Bias Formulas for Sensitivity Analysis of Unmeasured Confounding for General Outcomes, Treatments, and Confounders}. \textit{Epidemiology} 22 (1): 42.
\item Rosenbaum, Paul R. 2009. \href{http://www.tandfonline.com/doi/abs/10.1198/jasa.2009.tm08470}{Amplification of Sensitivity Analysis in Matched Observational Studies}. \textit{Journal of the American Statistical Association} 104 (488): 1398-1405.
%\item Paul Rosenbaum and Donald Rubin. 1983.  \href{http://www.jstor.org/stable/2345524} {Assessing Sensitivity to an Unobserved Binary Covariate in an Observational Study with Binary Outcome}. Journal of the Royal Statistical Society. Series B (Methodological) 45(2): 212-18.
%\item Hirano, Keisuke, Guido W. Imbens, Geert Ridder, and Donald B. Rubin. 2001. "Combining panel data sets with attrition and refreshment samples." Econometrica 69 (6).
\end{itemize}


\section{Causal Effects under Selection on Time-Invariant Characteristics}

\subsection{Difference-in-Differences Estimators}

\begin{itemize}
\item Identification, Estimation, Falsification tests
\end{itemize}

\emph{Readings: DID Theory}
\begin{itemize}
\item  Angrist and Pischke: Chapter 5.2-5.4 ($\star$)
\item Bertrand, Marianne, Esther Duflo, and Sendhil Mullainathan. 2004.  \href{http://qje.oxfordjournals.org/content/119/1/249.abstract} {How Much
Should We Trust Differences-in-Differences Estimates?} \textit{Quarterly Journal of
Economics} 119 (1): 249-275.
\end{itemize}

\emph{Readings: DID Applications}
\begin{itemize}
\item Lyall, Jason. 2009.  \href{http://www.jstor.org/stable/20684590} {Does Indiscriminate Violence Incite Insurgent Attacks?
Evidence from Chechnya}. \textit{Journal of Conflict Resolution} 53 (3): 331-62.
\item Card, David. 1990.  \href{http://www.jstor.org/stable/2523702} {The Impact of the Mariel Boatlift on the Miami Labor Market}, \textit{Industrial and Labor Relations Review} 44 (2): 245-257.
\item Card, David. and Alan B. Krueger. 1994.  \href{http://faculty.smu.edu/Millimet/classes/eco6352/papers/ck.pdf} {Minimum Wages and Employment: A Case Study
of the Fast-Food Industry in New Jersey and Pennsylvania}," \textit{American Economic
Review} 84 (4): 772-793.
\item Bechtel, Michael M. and Jens Hainmueller. 2011. \href{http://onlinelibrary.wiley.com/doi/10.1111/j.1540-5907.2011.00533.x/abstract} {How Lasting Is Voter Gratitude? An Analysis of the Short- and Long-Term Electoral Returns to Beneficial Policy}. \textit{American Journal of Political Science} 55 (4): 852-868.
\end{itemize}

\subsection{Panel Data Methods}

\begin{itemize}
\item Fixed Effects and Random Effects Estimation
\end{itemize}

\emph{Readings: Panel Methods Theory}
\begin{itemize}
\item  Angrist and Pischke: Chapter 5.1 ($\star$)
\item  Angrist and Pischke: Chapter 8 ($\star$)
\item Bai, Jushan. 2009. \href{http://www.jstor.org/stable/40263859}{Panel data models with interactive fixed effects}. \textit{Econometrica} 77(4): 1229-1279.
\end{itemize}

\emph{Readings: Panel Methods Applications}
\begin{itemize}
\item Ladd, Jonathan McDonald, and Gabriel S. Lenz. 2009.  \href{http://onlinelibrary.wiley.com/doi/10.1111/j.1540-5907.2009.00377.x/abstract} {Exploiting a Rare Communication Shift to Document the Persuasive Power of the News Media}. \textit{American Journal of Political Science} 53 (2): 394-410. ($\star$)
%\item Cox, Gary W., and William Terry. 2008.  %\href{http://dss.ucsd.edu/~wterry/index.html/research/cox_terry_2008.pdf} {Legislative Productivity in the 93d-105th Congresses}. \textit{Legislative Studies Quarterly} 33(4): 603-16.
\item Berrebi, Claude. and Esteban F. Klor. 2008.  \href{http://journals.cambridge.org/abstract_S0003055408080246} {Are Voters Sensitive to Terrorism? Direct Evidence from the Israeli Electorate}. \textit{American Political Science Review} 102 (3): 279-301.
\item Acemoglu, Daron, Simon Johnson, James A. Robinson, and Pierre Yared. 2008. \href{http://www.nber.org/papers/w11205}{Income and Democracy}. \textit{American Economic Review} 98 (3): 808-842.
\item Hainmueller, Jens and Hangartner, Dominik. 2016. \href{http://papers.ssrn.com/sol3/papers.cfm?abstract_id=2022064}{Does Direct Democracy Hurt Immigrant Minorities? Evidence from Naturalization Decisions in Switzerland}. \textit{American Journal of Political Science}.

\end{itemize}

\subsection{Synthetic Control Methods (Optional)}

\emph{Readings}
\begin{itemize}
\item Abadie, Diamond, and Hainmueller. 2010.  \href{http://www.tandfonline.com/doi/abs/10.1198/jasa.2009.ap08746#.Voip1jZh23I} {Synthetic Control Methods for Comparative Case Studies: Estimating the Effect of California's Tobacco Control Program}. \textit{Journal of the American Statistical Association} 105(490):  493-505.%($\star$)

\item Abadie, Diamond, and Hainmueller. 2014.  \href{https://web.stanford.edu/~jhain/Paper/AJPS2015a.pdf} {Comparative Politics and the Synthetic Control Method}. \textit{American Journal of Political Science}. 59(2): 495?510. %($\star$)

\item Abadie, Alberto and Javier Gardeazabal. 2003.  \href{http://pubs.aeaweb.org/doi/pdfplus/10.1257/000282803321455188} {The Economic Costs of Conflict: A Case Study of the Basque Country}. \textit{American Economic Review} 92 (1). 113-132.
\end{itemize}

\section{Causal Effects under Selection on Time-variant Characteristics}

\subsection{Instrumental Variables}

\begin{itemize}
\item Identification: Using Exogenous Variation in Treatment Intake Given by Instruments
\item Imperfect Compliance in Randomized Studies
\item Wald Estimator, Local Average Treatment Effects, 2SLS
\end{itemize}

\emph{Readings: Instrumental Variable Theory}
\begin{itemize}
\item  Angrist and Pischke: Chapter 4 ($\star$)
\item Morgan and Winship: Chapter 8
\item Morgan and Winship: Chapter 9 ($\star$)
%\item Angrist, Joshua D., and Alan B. Krueger. 2001.  \href{http://www.jstor.org/stable/2696517} {Instrumental Variables and the Search for Identification: From Supply and Demand to Natural Experiments}.
\item Angrist, Joshua D., Guido W. Imbens, and Donald B. Rubin. 1996. \href{http://www.jstor.org/stable/2291629}{Identification of Causal Effects Using Instrumental Variables.} \textit{Journal of the American Statistical Association} 91(434): 444-455.
\item Abadie, Alberto 2003. \href{http://www.hks.harvard.edu/fs/aabadie/gtep.pdf} {Semiparametric instrumental variable estimation of treatment response models}. Journal of Econometrics 113 (2003) 231-263.
\item Gerber, Alan S., and Donald P. Green. 2012. \textit{Field Experiments}. W. W. Norton. Chapters 5-6.
\item Sovey, Allison J. and Donald P. Green 2011. \href{http://faculty.smu.edu/millimet/classes/eco6374/papers/sovey\%20green\%202011.pdf}{Instrumental Variables Estimation in Political Science: A Readers’ Guide}. {\it American Journal of Political Science} 55 (1): 188-200.
\end{itemize}

\emph{Readings: Instrumental Variable Critique}
\begin{itemize}
\item Deaton, Angus. 2010. \href{http://pubs.aeaweb.org/doi/abs/10.1257/jel.48.2.424}{Instruments, Randomization, and Learning About Development}. \textit{Journal of Economic Literature} 48(2): 424-455.
\item Hernan, Miguel A., and James M. Robins. 2006. \href{http://www.jstor.org/stable/20486236?seq=1#page_scan_tab_contents}{Instruments for Causal Inference: An Epidemiologist's Dream?} \textit{Epidemiology} 17(4): 360-72.
\item Imbens, Guido W. 2010. \href{http://www.jstor.org/stable/20778730}{Better LATE Than Nothing: Some Comments on Deaton (2009) and Heckman and Urzua (2009)}. \textit{Journal of Economic Literature} 48(2): 399-423.
\end{itemize}

\emph{Readings: Instrumental Variable Applications}
\begin{itemize}
\item Holger L. Kern and Jens Hainmueller  \href{https://web.stanford.edu/~jhain/Paper/PA2009.pdf} {Opium for the Masses: How Foreign Free Media Can Stabilize Authoritarian Regimes}. Political Analysis (2009).
\item  Angrist and Krueger. 2001  \href{http://www.jstor.org/stable/2696517} {Instrumental Variables and the Search for Identification: From Supply and Demand to Natural Experiments}
\item Acemoglu, Daron, Simon Johnson, and James A. Robinson. 2001.  \href{http://www.jstor.org/stable/2677930} {The Colonial Origins of Comparative Development: An Empirical Investigation}. \textit{American Economic Review} 91(5): 1369-1401.
\item Clingingsmith, David, Asim Ijaz Khwaja, and Michael Kremer. 2009.  \href{http://qje.oxfordjournals.org/content/124/3/1133.short} {Estimating the Impact of the Hajj: Religion and Tolerance in Islam's Global Gathering}. \textit{Quarterly Journal of Economics} 124(3): 1133-1170.
\item Hidalgo, F. Daniel, Suresh Naidu, Simeon Nichter, and Neal Richardson. 2010. \href{http://www.mitpressjournals.org/doi/abs/10.1162/REST_a_00007}{Economic Determinants of Land Invasions}. \textit{Review of Economics and Statistics} 92(3): 505-523.
\item Angrist, Joshua D. 1990.  \href{http://www.jstor.org/stable/2006669} {Lifetime Earnings and the Vietnam Era Draft Lottery: Evidence from Social Security Administrative Records}. \textit{American
Economic Review} 80(3): 313-336.
\end{itemize}


\subsection{The Regression Discontinuity Design}

\begin{itemize}
\item Sharp and Fuzzy Designs, Identification, Estimation, Falsification Checks
\end{itemize}


\emph{Readings: RDD Theory}
\begin{itemize}
\item Imbens, Guido W., and Thomas Lemieux. 2008.  \href{http://linkinghub.elsevier.com/retrieve/pii/S0304407607001091} {Regression Discontinuity
Designs: A Guide to Practice}. \textit{Journal of Econometrics} 142 (2): 615-35. (Part
of special issue on RDD, all of which is of interest.) ($\star$)
\item  Angrist and Pischke: Chapter 6 ($\star$)
\item Hahn, Jinyong, Petra Todd and Wilbert Van der Klaauw. 2001.  \href{http://www.jstor.org/stable/2692190} {Identification and Estimation of Treatment Effects with a Regression Discontinuity Design}, \textit{Econometrica} 69 (1): 201-209.
\end{itemize}

\emph{Readings: RDD Applications}
\begin{itemize}
\item Eggers, Andrew, Fowler, Anthony, Hainmueller, Jens, Hall, Andrew B. and Snyder, James M. 2015. \href{http://onlinelibrary.wiley.com/doi/10.1111/ajps.12127/abstract}{On the Validity of the Regression Discontinuity Design for Estimating Electoral Effects: New Evidence from over 40,000 Close Races}. \textit{American Journal of Political Science}  59(1): 259-274 ($\star$).
%\item Ferraz, C., and F. Finan. 2008.  \href{http://nbn-resolving.de/urn:nbn:de:101:1-2008032608} {Motivating Politicians: The Impacts of Monetary Incentives on Quality and Performance}. Mimeo. 2009 NBER Working paper w14906.
\item Hidalgo, F. Daniel. 2012. Fraud or Enfranchisement? The Consequences of Electronic Voting for Political Representation in Brazil. Working Paper.
\item Lee, David S. 2008.  \href{http://linkinghub.elsevier.com/retrieve/pii/S0304407607001121} {Randomized Experiments from Non-random Selection in U.S. House Elections}. \textit{Journal of Econometrics} 142 (2): 675-697.  ($\star$)
%\item  Eggers and Hainmueller:  \href{http://www.mit.edu/~jhainm/Paper/mpfs.pdf} {The Value of Political Power: Estimating Returns to Office in Post-War British Politics}.
%\item Butler, Daniel M., and Matthew J. Butler. 2006.  \href{http://pan.oxfordjournals.org/cgi/content/full/14/4/439?ijkey=RxlTLbaIFPaC8T9&keytype=ref} {Splitting the Difference? Causal Inference and Theories of Split-Party Delegations}. \textit{Political Analysis} 14(4): 439-55.
\item Hainmueller, Jens, and Holger Lutz Kern. 2008. \href{http://www.sciencedirect.com/science/article/pii/S0261379407000996} {Incumbency as a Source of Spillover Effects in Mixed Electoral Systems: Evidence from a Regression-
Discontinuity Design}. \textit{Electoral Studies} 27: 213-27.
\item Caughey, Devin, and Jasjeet Sekhon. 2011. \href{http://pan.oxfordjournals.org/content/19/4/385.abstract}{Elections and the Regression Discontinuity Design: Lessons From Close U.S. House Races, 1942-2008}. \textit{Political Analysis} 19 (4): 385-408.
\item Eggers, Andrew, Freier, Ronny, Grembi, Veronica and Nannicini, Tommaso. 2016. \href{http://andy.egge.rs/papers/EggersFreierGrembiNannicini_pop_thresholds.pdf}{Regression Discontinuity Designs Based on Population Thresholds: Pitfalls and Solutions}.
\item Calonico, Sebastian, Cattaneo, Matias, and Titiunik, Rocio. 2014. \href{http://onlinelibrary.wiley.com/doi/10.3982/ECTA11757/abstract}{Robust Nonparametric Confidence Intervals for Regression-Discontinuity Designs} \textit{Econometrica} 82(6): 2295-2326.
\item Hainmueller, Jens, Hall, Andrew, and Snyder, James. 2015. \href{http://www.jstor.org/stable/10.1086/681238}{Assessing the External Validity of Election RD Estimates: An Investigation of the Incumbency Advantage} \textit{Journal of Politics}. 77(3): 707-720.
\item Hainmueller, Jens, Hangartner, Dominik, and Pietrantuono, Giuseppe. 2015. \href{http://www.pnas.org/content/112/41/12651.abstract?sid=8efc8281-95c9-4973-9a2c-bae8df47fd4a}{Naturalization Fosters the Long-Term Political Integration of Immigrants}. \textit{Proceedings of the National Academy of Sciences}. 112 (41) 12651-12656.
\item Bertanha M, Imbens G. \href{http://www.nber.org/papers/w20773}{External Validity in Fuzzy Regression Discontinuity Designs}. 2014. Working Paper.

\end{itemize}

\section{Distributional Effects (Optional)}

\subsection{Quantile Regression}

\emph{Readings}
\begin{itemize}
\item Angrist and Pischke: Chapter 7 ($\star$)
\item Koenker, Roger and Kevin F. Hallock. 2001.  \href{http://www.jstor.org/stable/2696522} {Quantile Regression}. \textit{Journal of Economic Perspectives} 15 (4): 143-156($\star$)
\item Bruenig, Christian and Brian D. Jones. 2011. \href{http://pan.oxfordjournals.org/content/19/1/103.full.pdf+html}{Stochastic Process Methods with an Application to Budgetary Data}. \textit{Political Analysis} 19 (1): 103-117
\end{itemize}

\subsection{Distributional Effects in Difference-in-Differences}

\emph{Readings}
\begin{itemize}
\item Athey, Susan and Guido W. Imbens. 2006.  \href{http://www.jstor.org/stable/3598807} {Identification and Inference in Nonlinear Difference-in-Differences Models}. \textit{Econometrica} 74 (2): 431--497.($\star$)
\end{itemize}


\subsection{Instrumental Variables for Quantile Effects}
%
\emph{Readings}
\begin{itemize}
\item Abadie, Alberto, Joshua Angrist, and Guido Imbens. 2002. \href{http://www.hks.harvard.edu/fs/aabadie/qtep.pdf} {Instrumental Variables Estimates of the Effect of Subsidized Training on the Quantiles of Trainee Earnings}. \textit{Econometrica} 70 (1): 91--117.

\end{itemize}

\clearpage
\section*{Model Based Inference}
\setcounter{section}{0}

\subsection*{Books}
\begin{itemize}
\item[-] Agresti, Alan. 2015.  \emph{Foundations of Linear and Generalized Linear Models.} Wiley. (Hereafter AA)
\item[-] Wasserman, Larry. 2013.  \emph{All of Statistics: A Concise Course in Statistical Inference}. Springer. (Hereafter AS) 
\item[-] Wasserman, Larry. 2006.  \emph{All of Nonparametric Statistics} (Hereafter ANS)
\item[-] Degroot, Morris and Mark Schervish. Probability and Statistics
\item[-] Bertsekas, Dimitri P and Tsitsiklis, John.  Introduction to Probability Theory
\item[-] Hastie, Tibshirani, and Friedman.  2009.  \emph{The Elements of Statistical Learning: Data Mining, Inference, and Prediction} 2nd edition.  Springer.  (Hereafter ES)
\end{itemize}


\section{Likelihood Theory of Inference}
\begin{itemize}
\item Likelihood curve/interpretation
\item Asymptotic property of MLE estimates
\item Invariance, Cramer-Rao Lower Bound
\end{itemize}

\textit{Required Readings}
\begin{itemize}
\item AS Chapter 9
\item AA 4.1-4.4
\item Degroot and Schervish 6.1-6.5 
\end{itemize}

\section{Generalized Linear Models}
\begin{itemize}
\item Models for normally distributed outcomes
\item Probit/Logit for binary choice
\item Ordered probit, multinomial probit and multinomial logit
\item Event count and survival analysis models
\item Measures of model fit 
\item Likelihood ratio test and Wald Test
\end{itemize}
\textit{Required Readings}
\begin{itemize}
\item AA Chapter 4.4-4.6, 5, 6.1, 6.2, and 7
\item AS Chapter 10.1-10.3, 10.6, 10.8
\end{itemize}

\section{Machine Learning Methods}
\begin{itemize}
\item LASSO
\item Ridge
\item Boosting/Bagging
\end{itemize}

\textit{Required Readings}
\begin{itemize}
\item[-] Elements of Statistical Learning, 3.4.1-3.4.2
\item[-] Hainmueller, Jens and Chad Hazlett. 2014.  ``Kernel Regularized Least Squares: Reducing Misspecification Bias with a Flexible and Interpretable Machine Learning Approach" \emph{Political Analysis}. 22, 2. 143-168. 
\end{itemize}

\section{Nonparametric Estimation}

\begin{itemize}
\item Density estimation
\item Bandwidth selection 
\item nonparametric regression
\end{itemize}


\begin{itemize}
\item[-] AS Chapter 6, 20
\item[-] ANS Chapter 6
\item[-] Beck, Nathaniel and Simon Jackman. 1998. ``Beyond Linearity by Default: Generalized Additive Models". \emph{American Journal of Political Science} 42, 2. 596-627.
\end{itemize}


\end{document}












